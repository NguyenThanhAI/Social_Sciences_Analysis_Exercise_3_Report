\documentclass[14pt, a4paper]{article}
\usepackage{minitoc}
\usepackage[left=3.00cm, right=2.5cm, top=2.00cm, bottom=2.00cm]{geometry}
\usepackage{amsmath}
\usepackage{amssymb}
\usepackage{amsthm}
\usepackage{mathtools}
\usepackage{graphicx}
%\usepackage{algpseudocode}
%\usepackage{algorithm}
\usepackage[ruled,vlined,linesnumbered]{algorithm2e}
\usepackage{blindtext}
\usepackage{setspace}
\usepackage[utf8]{inputenc}
\usepackage[utf8]{vietnam}
\usepackage[center]{caption}
\usepackage[shortlabels]{enumitem}
\usepackage{fancyhdr} % header, footer
\usepackage{hyperref} % loại bỏ border với mục lục và công thức
\usepackage[nonumberlist, nopostdot, nogroupskip]{glossaries}
\usepackage{glossary-superragged}
\usepackage{tikz,tkz-tab}
\usepackage{pythonhighlight}
\setglossarystyle{superraggedheaderborder}
\pagestyle{fancy}
%\usepackage[style=numeric,sortcites]{biblatex}
%\addbibresource{ref.bib}
%\usepackage[numbers]{natbib}
\usepackage{indentfirst}
\usepackage[natbib,backend=biber,style=ieee, sorting=ynt]{biblatex}

\usepackage{caption}
\usepackage{subcaption}

\bibliography{ref.bib}

\graphicspath{{./figures/}}

\fancyhf{}
%\rhead{\textbf{Môn học: Các phương pháp thống kê hiện đại trong nghiên cứu Xã hội học}}
\lhead{\textbf{GVHD: TS. Trịnh Quốc Anh}}
\rfoot{\thepage}
\lfoot{\textbf{Học viên thực hiện: Nguyễn Chí Thanh - 21007925}}
\renewcommand{\headrulewidth}{0.4pt}
\renewcommand{\footrulewidth}{0.4pt}
%
%\numberwithin{equation}{section}
%\numberwithin{algorithm}{section}
%\numberwithin{figure}{section}
%
%\setlength{\parindent}{0.5cm}
%
%\setcounter{secnumdepth}{3} % Cho phép subsubsection trong report
%\setcounter{tocdepth}{3} % Chèn subsubsection vào bảng mục lục

%\newtheorem{dl}{Định lý}
%\newtheorem{md}{Mệnh đề}
%\newtheorem{bd}{Bổ đề}
%\newtheorem{dn}{Định nghĩa}
%\newtheorem{hq}{Hệ quả}

%\newtheorem{baitap}{Bài tập}
%\newtheorem*{loigiai}{Lời giải}

%\numberwithin{dl}{section}
%\numberwithin{md}{section}
%\numberwithin{bd}{section}
%\numberwithin{dn}{section}
%\numberwithin{hq}{section}

\setlength{\parindent}{0cm}

\newtheorem{dl}{Định lý}
\newtheoremstyle{sltheorem}
{}                % Space above
{}                % Space below
{\normalfont}        % Theorem body font % (default is "\upshape")
{}                % Indent amount
{\bfseries}       % Theorem head font % (default is \mdseries)
{.}               % Punctuation after theorem head % default: no punctuation
{ }               % Space after theorem head
{}                % Theorem head spec
\theoremstyle{sltheorem}
\newtheorem{baitap}{Bài tập}
\newtheoremstyle{soltheorem}
{}                % Space above
{}                % Space below
{\normalfont}        % Theorem body font % (default is "\upshape")
{}                % Indent amount
{\bfseries}       % Theorem head font % (default is \mdseries)
{.}               % Punctuation after theorem head % default: no punctuation
{\newline}               % Space after theorem head
{}                % Theorem head spec
\theoremstyle{soltheorem}
\newtheorem*{loigiai}{Lời giải}

\onehalfspacing

\begin{document}
\begin{titlepage}

    \newcommand{\HRule}{\rule{\linewidth}{0.5mm}} % Defines a new command for the horizontal lines, change thickness here

    \center % Center everything on the page

    %----------------------------------------------------------------------------------------
    %	HEADING SECTIONS
    %----------------------------------------------------------------------------------------
    \textsc{\LARGE Đại học Quốc Gia Hà Nội}\\[0.5cm]
    \textsc{\LARGE Trường đại học Khoa học tự nhiên}\\[0.5cm] % Name of your university/college
    \textsc{\LARGE Khoa Toán - Cơ - Tin học}\\[0.5cm]

    \includegraphics[scale=0.2]{HUS-logo.jpg}\\[0.5cm]

    \textsc{\Large Chuyên ngành: Khoa học dữ liệu}\\[0.5cm] % Major heading such as course name


    %----------------------------------------------------------------------------------------
    %	TITLE SECTION
    %----------------------------------------------------------------------------------------

    \HRule \\[0.4cm]
    { \huge \bfseries Bài tập môn học}\\[0.4cm] % Title of your document
    \HRule \\[1.5cm]

    \textsc{\Large Môn học: Các phương pháp thống kê hiện đại \\ trong nghiên cứu Xã hội học}\\[1cm] % Minor heading such as course title


    \textsc{\Large Bài tập số 3}\\[1cm]


    %----------------------------------------------------------------------------------------
    %	AUTHOR SECTION
    %----------------------------------------------------------------------------------------
    \begin{minipage}{0.4\textwidth}
        \begin{flushleft} \large
        \emph{Giảng viên hướng dẫn:} \\
        TS. Trịnh Quốc Anh % Supervisor's Name
        \end{flushleft}
    \end{minipage}\\[0.5cm]

    \begin{minipage}{0.4\textwidth}
    \begin{flushleft} \large
    \emph{Học viên thực hiện:}\\
    Nguyễn Chí Thanh \\
    MSHV: 21007925 \\ % Your name
    Lớp: Khoa học dữ liệu - K4
    \end{flushleft}
    \end{minipage}


    % If you don't want a supervisor, uncomment the two lines below and remove the section above
    %\Large \emph{Author:}\\
    %John \textsc{Smith}\\[3cm] % Your name

    %----------------------------------------------------------------------------------------
    %	DATE SECTION
    %----------------------------------------------------------------------------------------

    % I don't want day because it is English
    % {\large \today}\\[2cm] % Date, change the \today to a set date if you want to be precise

    %----------------------------------------------------------------------------------------
    %	LOGO SECTION
    %----------------------------------------------------------------------------------------

    %\includegraphics{logo/rsz_3logo-khtn.png}\\[1cm] % Include a department/university logo - this will require the graphicx package

    %----------------------------------------------------------------------------------------

    \vfill % Fill the rest of the page with whitespace

\end{titlepage}

\nocite{*}

\newpage

\begin{baitap}
    Áp dụng các phương pháp Ridge, LASSO, và PCA cho dữ liệu Sonar.

    Hãy so sánh các kết quả thu được và biện luận để đưa ra mô hình phù hợp nhất.
\end{baitap}

\begin{loigiai}
    Hàm mất mát của mô hình logistic regression:
    \begin{equation*}
        \mathcal{L} = -\dfrac{1}{N}\sum_{i=1}^N (y_i \log \hat{y}_i + (1-y_i)\log(1-\hat{y}_i)) 
    \end{equation*}

    với $\hat{y}=p(y=1\vert \bold{x})=f(\bold{x}) = \sigma(w_0 + w_1 x_1 + w_2 x_2 + \dots w_n x_n) = \sigma(\bold{w}^T \bold{x})$ là kết quả đầu ra của mô hình logistic tương ứng với đầu vào $\bold{x}$.

    Hàm mất mát của phương pháp Ridge:

    \begin{equation*}
        \mathcal{L}_{\mathrm{Ridge}} = -\dfrac{1}{N}\sum_{i=1}^N (y_i \log \hat{y}_i + (1-y_i)\log(1-\hat{y}_i)) + \lambda \lVert \bold{w} \rVert_2^2
    \end{equation*}

    Hàm mất mát của phương pháp LASSO:

    \begin{equation*}
        \mathcal{L}_{\mathrm{Ridge}} = -\dfrac{1}{N}\sum_{i=1}^N (y_i \log \hat{y}_i + (1-y_i)\log(1-\hat{y}_i)) + \lambda \lVert \bold{w} \rVert_1
    \end{equation*}

    Ta sử dụng 5 mô hình đã được sử dụng trong bài tập trước:
    
\end{loigiai}

\newpage
\printbibliography[title={TÀI LIỆU THAM KHẢO}]

\newpage
\appendix

\section{Các thuật toán tối ưu hóa}


\begin{algorithm}[h!]
    \DontPrintSemicolon
    \KwIn{độ dài bước $\lbrace \alpha_t \rbrace_{t=1}^{T}$, $w_0$ khởi tạo, hàm mục tiêu $\ell(w)$}
    \KwOut{$w$ đã được học}

    \For{$t \gets 1$ \KwSty{to} $T$} {
        $g_t \gets \nabla \ell(w_{t-1})$\;
        $w_t \gets w_{t-1} - \alpha_t g_t$
    }
    \Return{$w_T$}\;
    \caption{Thuật toán Gradient Descent}
\end{algorithm}


\begin{algorithm}[h!]
    \DontPrintSemicolon
    \KwIn{độ dài bước $\lbrace \alpha_t \rbrace_{t=1}^{T}$, hệ số $\beta_1$, $w_0$ khởi tạo, hàm mục tiêu $\ell(w)$}
    \KwOut{$w$ đã được học}
    $m_0 \gets 0$\;
    \For{$t \gets 1$ \KwSty{to} $T$} {
        $g_t \gets \nabla \ell(w_{t-1})$\;
        $m_t \gets \beta_1 m_{t-1} + (1-\beta_1) g_t$\;
        $w_t \gets w_{t-1} - \alpha_t m_t$\;
    }
    \Return{$w_T$}\;
    \caption{Thuật toán Momentum}
\end{algorithm}


\begin{algorithm}[h!]
    \KwIn{độ dài bước $\lbrace \alpha_t \rbrace_{t=1}^{T}$, $w_0$ khởi tạo, hàm mục tiêu $\ell(w)$}
    \KwOut{$w$ đã được học}

    $v_0 \gets 0$\;
    \For{$t \gets 1$ \KwSty{to} $T$} {
        $g_t \gets \nabla \ell(w_{t-1})$\;
        $v_t \gets v_{t-1} + g_t^2$\;
        $w_t \gets w_{t-1} - \alpha_t \dfrac{g_t}{\sqrt{v_t} + \epsilon}$\;
    }

    \Return{$w_T$}\;
    \caption{Thuật toán Adagrad}
\end{algorithm}

\begin{algorithm}[h!]
    \DontPrintSemicolon
    \KwIn{độ dài bước $\lbrace \alpha_t \rbrace_{t=1}^{T}$, hệ số $\beta$, $w_0$ khởi tạo, hàm mục tiêu $\ell(w)$}
    \KwOut{$w$ đã được học}

    $v_0 \gets 0$\;
    $d_0 \gets 0$\;

    \For{$t \gets 1$ \KwSty{to} $T$} {
        $g_t \gets \nabla \ell (w_{t-1})$\;
        $v_t \gets \beta v_{t-1} + (1-\beta) g_t^2$\;
        $\Delta w \gets -\alpha_t \dfrac{\sqrt{d_{t-1} + \epsilon}g_t}{\sqrt{v_t + \epsilon}}$\;
        $w_t \gets w_{t-1} + \delta w$\;
        $d_t \gets \beta d_{t-1} + (1-\beta) \delta w^2$\;
    }

    \Return{$w_T$}\;
    \caption{Thuật toán Adadelta}
\end{algorithm}


\begin{algorithm}[h!]
    \DontPrintSemicolon
    \KwIn{độ dài bước $\lbrace \alpha_t \rbrace_{t=1}^{T}$, hệ số $\beta$, $w_0$ khởi tạo, hàm mục tiêu $\ell(w)$}
    \KwOut{$w$ đã được học}

    $v_0 \gets 0$\;
    \For{$t \gets 1$ \KwSty{to} $T$} {
        $g_t \gets \nabla \ell (w_{t-1})$\;
        $v_t \gets \beta v_{t-1} + (1-\beta) g_t^2$\;
        $w_t \gets w_{t-1} - \alpha_t \dfrac{g_t}{\sqrt{v_t} + \epsilon}$\;
    }

    \Return{$w_T$}\;
    \caption{Thuật toán RMSProp}
\end{algorithm}


\begin{algorithm}[h!]
    \DontPrintSemicolon
    \KwIn{độ dài bước $\lbrace \alpha_t \rbrace_{t=1}^{T}$, các hệ số $\beta_1, \beta_2$, $w_0$ khởi tạo, hàm mục tiêu $\ell(w)$}
    \KwOut{$w$ đã được học}

    $m_0 \gets 0$\;
    $v_0 \gets 0$\;

    \For{$t \gets 1$ \KwSty{to} $T$} {
        $g_t \gets \nabla \ell(w_{t-1})$\;
        $m_t \gets \beta_1 m_{t-1} + (1 - \beta_1)g_t$\;
        $v_t \gets \beta_2 v_{t-1} + (1 - \beta_2)g_t^2$\;
        $\hat{m}_t \gets \dfrac{m_t}{1 - \beta_1^t}$\;
        $\hat{v}_t \gets \dfrac{v_t}{1 - \beta_2^t}$\;
        $w_t \gets w_{t-1} - \alpha_t \dfrac{\hat{m}_t}{\sqrt{\hat{v}_t} + \epsilon}$\;
    }

    \Return{$w_T$}\;
    \caption{Thuật toán Adam}
\end{algorithm}


\begin{algorithm}[h!]
    \DontPrintSemicolon
    \KwIn{độ dài bước $\lbrace \alpha_t \rbrace_{t=1}^{T}$, các hệ số $\beta_1, \beta_2$, $w_0$ khởi tạo, hàm mục tiêu $\ell(w)$}
    \KwOut{$w$ đã được học}

    $m_0 \gets 0$\;
    $u_0 \gets 0$\;

    \For{$t \gets 1$ \KwSty{to} $T$} {
        $g_t \gets \nabla \ell(w_{t-1})$\;
        $m_t \gets \beta_1 m_{t-1} + (1 - \beta_1)g_t$\;
        $u_t \gets \max (\beta u_{t-1}, \lVert g_t \rVert_{\infty})$\;
        $w_t \gets w_{t-1} - \alpha_t \dfrac{m_t}{(1-\beta_1^t)u_t + \epsilon}$\;
    }

    \Return{$w_T$}\;
    \caption{Thuật toán Adamax}
\end{algorithm}


\begin{algorithm}[h!]
    \DontPrintSemicolon
    \KwIn{độ dài bước $\lbrace \alpha_t \rbrace_{t=1}^{T}$, các hệ số $\beta_1, \beta_2$, $w_0$ khởi tạo, hàm mục tiêu $\ell(w)$}
    \KwOut{$w$ đã được học}

    $m_0 \gets 0$\;
    $v_0 \gets 0$\;

    \For{$t \gets 1$ \KwSty{to} $T$} {
        $g_t \gets \nabla \ell(w_{t-1})$\;
        $m_t \gets \beta_1 m_{t-1} + (1 - \beta_1)g_t$\;
        $v_t \gets \beta_2 v_{t-1} + (1 - \beta_2)g_t^2$\;
        $\hat{v}_t \gets \max(\hat{v}_{t-1}, v_t)$\;
        $w_t \gets w_{t-1} - \alpha_t \dfrac{m_t}{\sqrt{\hat{v}_t} + \epsilon}$\;
    }

    \Return{$w_T$}\;
    \caption{Thuật toán AMSGrad}
\end{algorithm}


\begin{algorithm}[h!]
    \DontPrintSemicolon
    \KwIn{độ dài bước $\lbrace \alpha_t \rbrace_{t=1}^{T}$, các hệ số $\beta_1, \beta_2$, $w_0$ khởi tạo, hàm mục tiêu $\ell(w)$}
    \KwOut{$w$ đã được học}

    $m_0 \gets 0$\;
    $v_0 \gets 0$\;

    \For{$t \gets 1$ \KwSty{to} $T$} {
        $g_t \gets \nabla \ell(w_{t-1})$\;
        $m_t \gets \beta_1 m_{t-1} + (1 - \beta_1)g_t$\;
        $v_t \gets \beta_2 v_{t-1} + (1 - \beta_2)g_t^2$\;
        $\hat{m}_t \gets \dfrac{m_t}{1 - \beta_1^t}$\;
        $\hat{v}_t \gets \dfrac{v_t}{1 - \beta_2^t}$\;
        $w_t \gets w_{t-1} - \dfrac{\alpha_t}{\sqrt{\hat{v}_t} + \epsilon} \Bigg( \beta_1 \hat{m}_t + \dfrac{1 - \beta_1}{1 - \beta_1^t}g_t \Bigg)$\;
    }

    \Return{$w_T$}\;
    \caption{Thuật toán Nadam}
\end{algorithm}


\begin{algorithm}[h!]
    \DontPrintSemicolon
    \KwIn{độ dài bước $\lbrace \alpha_t \rbrace_{t=1}^{T}$, các hệ số $\beta_1, \beta_2$, $w_0$ khởi tạo, hàm mục tiêu $\ell(w)$}
    \KwOut{$w$ đã được học}

    $m_0 \gets 0$\;
    $s_0 \gets 0$\;

    \For{$t \gets 1$ \KwSty{to} $T$} {
        $g_t \gets \nabla \ell(w_{t-1})$\;
        $m_t \gets \beta_1 m_{t-1} + (1 - \beta_1)g_t$\;
        $s_t \gets \beta_2 s_{t-1} + (1-\beta_2)(g_t - m_t)^2 + \epsilon$\;
        $\hat{m}_t \gets \dfrac{m_t}{1 - \beta_1^t}$\;
        $\hat{s}_t \gets \dfrac{s_t}{1 - \beta_2^2}$\;
        $w_t \gets w_{t-1} - \alpha_t \dfrac{\hat{m}_t}{\sqrt{\hat{s}_t} + \epsilon}$
    }

    \Return{$w_T$}\;
    \caption{Thuật toán AdaBelief}
\end{algorithm}


\begin{algorithm}[h!]
    \DontPrintSemicolon
    \KwIn{độ dài bước $\lbrace \alpha_t \rbrace_{t=1}^{T}$, các hệ số $\beta_1, \beta_2$, $w_0$ khởi tạo, hàm mục tiêu $\ell(w)$}
    \KwOut{$w$ đã được học}
    $m_0 \gets 0$\;
    $v_0 \gets 0$\;
    $\rho_{\infty} \gets 2/(1-\beta_2)-1$\;
    \For{$t \gets 1$ \KwSty{to} $T$}{
        $g_t \gets \nabla \ell(w_{t-1})$\;
        $m_t \gets \beta_1 m_{t-1} + (1 - \beta_1)g_t$\;
        $v_t \gets \beta_2 v_{t-1} + (1 - \beta_2)g_t^2$\;
        $\widehat{m}_t \gets m_t / (1 - \beta_1^t)$\;
        $\rho_t \gets \rho_{\infty} - 2t\beta_2^t / (1 - \beta_2^t)$\;
        \If {$\rho_t < 4$} {
            $\widehat{v}_t \gets \sqrt{v_t / (1 - \beta_2^t)}$\;
            $r_t \gets \sqrt{\frac{(\rho_t - 4)(\rho_t - 2)\rho_{\infty}}{(\rho_{\infty} - 4)(\rho_{\infty} - 2)\rho_t}}$\;
            $w_t \gets w_{t-1} - \alpha_t r_t \widehat{m}_t / (\widehat{v}_t + \epsilon)$\;
        } \Else {
            $w_t \gets w_{t-1} - \alpha_t \widehat{m}_t$\;
        }
    }
    \Return{$w_T$}\;
    \caption{Thuật toán RAdam}
\end{algorithm}


\begin{algorithm}[h!]
    \DontPrintSemicolon
    \KwIn{độ dài bước $\lbrace \alpha_t \rbrace_{t=1}^{T}$, hệ số $\beta_1$, $w_0$ khởi tạo, hàm mục tiêu $\ell(w)$}
    \KwOut{$w$ đã được học}
    $m_0 \gets 0$\;
    $v_0 \gets 0$\;

    \For{$t \gets 1$ \KwSty{to} $T$}{
        $g_t \gets \nabla \ell(w_{t-1})$\;
        $m_t \gets \beta_1 m_{t-1} + (1 - \beta_1)g_t$\;
        $\eta_t \gets \dfrac{1}{\sqrt{v_{t-1}} + \epsilon}$\;
        $w_t \gets w_{t-1} - \alpha_t \dfrac{\eta_t}{\lVert \eta_t / \sqrt{d} \rVert_2 \odot m_t}$\;
        $v_t \gets \beta_2 v_{t-1} + (1-\beta_2)g_t^2$\;
    }
    \Return{$w_T$}\;
    \caption{Thuật toán Avagrad}
\end{algorithm}

\end{document}